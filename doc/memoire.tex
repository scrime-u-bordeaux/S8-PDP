\documentclass{article}

\usepackage[utf8]{inputenc} % accents
\usepackage[T1]{fontenc}      % caractères français
\usepackage{geometry}         % marges
\usepackage[francais]{babel}  % langue
\usepackage{graphicx}         % images
\usepackage{verbatim}
\usepackage{url}
\bibliographystyle{alpha}

\title{Apprendre et comprendre l'improvisation musicale}
\subtitle{Mémoire}
\author{Alexandre Casanova-Franger\\
  \and
  Gauthier Lamarque\\
  \and
  Paul Simorre\\
  \and
  Lucas Vivas\\}

\begin{document}
\malketitle
\section{Introduction}
\paragraph{}
Ce projet a été réalisé par quatre étudiants de première année de Master Informatique de l'Université de Bordeaux (spécialité Génie Logiciel), dans le cadre de l'UE Projet de Programmation.

\subsection{Un outil pour assister l'improvisation musicale}
Le projet a été proposé par Myriam Desainte-Catherine, chercheur au Studio de Création et de Recherche en Informatique et Musiques Expérimentales (SCRIME) et enseignante à l'ENSEIRB-MATMECA, et Yacine Amarouchene, physicien du Laboratoire d'Onde et Matière d'Aquitaine (LOMA). Il a été pensé comme un outil d'aide à \"l'improvisation musicale\", un dispositif permettant d'aider un groupe de musiciens à improviser mais permettant aussi de mieux comprendre la nature de l'improvisation en musique.

\subsection{Improvisation et corrélation}
\paragraph{}
Il est essentiel de comprendre ces deux notions afin de bien situer l'intérêt du projet ; elles sont le c\oe ur de notre travail. \\
Lors d'une improvisation en groupe, des musiciens jouent sans règle établie ; ils cherchent alors à fournir à leur auditoire une production cohérente, ou du moins au sein de laquelle chaque musicien joue un rôle dans l'harmonie du morceau en s'accordant avec ses pairs. La relation qui unit le jeu de deux musiciens et évalue la qualité de leur accord porte un nom : c'est la corrélation. \\
Au sens premier et basique du terme, la corrélation et le rapport réciproque entre deux éléments. En statistiques, on parle souvent de corrélation pour mesurer l'intensité de la liaison existant entre deux variables. Dans le cadre de l'étude de signaux sonores, les traductions graphiques de ces signaux par transformée de Fourier peuvent jouer le rôle des courbes dont on mesure la corrélation. En musique, il n'existe pas une mais un nombre indéfini de \"fonctions de corrélation\"potentiellement existantes, dont la complexité et les paramètres varient. \" Comprendre\" la corrélation et l'improvisation, dans le cadre de ce projet, c'est aussi trouver, inventer la fonction de corrélation qui répond le mieux aux besoins d'un groupe d'improvisateurs. Une bonne fonction de corrélation, dans ce cadre spécifique, est une fonction qui fait en sorte qu'un groupe de musiciens joue un morceau plus satisfaisant lorsque ses membres sont plus corrélés. \\

\subsection{L'existant}
\paragraph{}
Ce projet a été initié par les clients précédemment cités il y a plus d'un an. Nous sommes le troisième groupe à travailler sur ce projet et développons donc sur la base des travaux réalisés successivement par nos prédécesseurs.

\end{document}
