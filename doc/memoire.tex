\documentclass{article}

\usepackage[utf8]{inputenc} % accents
\usepackage[T1]{fontenc}      % caractères français
\usepackage{geometry}         % marges
\usepackage[francais]{babel}  % langue
\usepackage{graphicx}         % images
\usepackage{verbatim}
\usepackage{url}
\bibliographystyle{alpha}

\title{Apprendre et comprendre l'improvisation musicale}
\subtitle{Mémoire}
\author{Alexandre Casanova-Franger\\
  \and
  Gauthier Lamarque\\
  \and
  Paul Simorre\\
  \and
  Lucas Vivas\\}

\begin{document}
\malketitle
\section{Introduction}
\paragraph{}
Ce projet a été réalisé par quatre étudiants de première année de
Master Informatique de l'Université de Bordeaux (spécialité Génie
Logiciel), dans le cadre de l'UE Projet de Programmation.

\subsection{Un outil pour assister l'improvisation musicale}
Le projet a été proposé par Myriam Desainte-Catherine, chercheur au
Studio de Création et de Recherche en Informatique et Musiques
Expérimentales (SCRIME) et enseignante à l'ENSEIRB-MATMECA, et Yacine
Amarouchene, physicien du Laboratoire d'Onde et Matière d'Aquitaine
(LOMA). Il a été pensé comme un outil d'aide à \"l'improvisation
musicale\", un dispositif permettant d'aider un groupe de musiciens à
improviser mais permettant aussi de mieux comprendre la nature de
l'improvisation en musique.

\subsection{Improvisation et corrélation}
\paragraph{}
Il est essentiel de comprendre ces deux notions afin de bien situer
l'intérêt du projet ; elles sont le c\oe ur de notre travail. \\ Lors
d'une improvisation en groupe, des musiciens jouent sans règle établie
; ils cherchent alors à fournir à leur auditoire une production
cohérente, ou du moins au sein de laquelle chaque musicien joue un
rôle dans l'harmonie du morceau en s'accordant avec ses pairs. La
relation qui unit le jeu de deux musiciens et évalue la qualité de
leur accord porte un nom : c'est la corrélation. \\
Au sens premier et basique du terme, la corrélation est le rapport
réciproque entre deux éléments. En statistiques, on parle souvent de
corrélation pour mesurer l'intensité de la liaison existant entre deux
variables. Dans le cadre de l'étude de signaux sonores, les
traductions graphiques de ces signaux par transformée de Fourier
peuvent jouer le rôle des courbes dont on mesure la corrélation. \\
En musique, il n'existe pas une mais un nombre indéfini de \"fonctions
de corrélation\" potentiellement existantes, dont la complexité et les
paramètres varient. \" Comprendre\" la corrélation et l'improvisation,
dans le cadre de ce projet, c'est aussi trouver, inventer la fonction
de corrélation qui répond le mieux aux besoins d'un groupe
d'improvisateurs. Une bonne fonction de corrélation, dans ce cadre
spécifique, est une fonction qui fait en sorte qu'un groupe de
musiciens joue un morceau plus satisfaisant lorsque ses membres sont
davantage corrélés. \\

\subsection{L'existant}
\paragraph{}
Ce projet a été initié par les clients précédemment cités il y a plus
d'un an. Nous sommes le troisième groupe à travailler sur ce projet et
développons donc sur la base des travaux réalisés successivement par
nos prédécesseurs.

\subsubsection{Premiers travaux réalisés sur le projet}
Les premiers travaux portant sur ce projet ont été réalisés par un
groupe de six étudiants de l'ENSEIRB-MATMECA. Cette équipe est la
première à réaliser un programme informatique analysant et comparant
des pistes mono-instrumentales pour évaluer leurs corrélations. Ce
programme, prenant en entrées des pistes sonores, retourne une matrice
graphique prenant les mêmes pistes en abscisses et en ordonnées et
retournant pour chaque case une couleur permettant d'évaluer la
corrélation existant entre les deux pistes correspondantes grâce à un
code précis. \\
\\
MATRICEENSEIRB
\\
Ci-dessus, la matrice obtenue à l'issue d'un test du programme. Plus
la couleur d'une case est proche du vert, plus la corrélation entre
les deux pistes correspondant au point d'abscisse et au point
d'ordonnée est élevée. À l'inverse, une case dont la couleur est
proche du rouge indique que les deux pistes évaluées sont
décorrélées. La corrélation d'une piste mono-instrumentale avec
elle-même, qui donne toujours un résultat maximal, n'est pas calculé,
ce qui se traduit sur le résultat graphique ci-dessus par une
diagonale de cases noires. La matrice de corrélation est donc
systématiquement symétrique.

\subsubsection{Le système embarqué BELA}
Avant de confier la suite du projet à d'autres étudiants, nos clients
ont choisi de le rendre plus fonctionnel grâce à l'emploi d'un système
embarqué particulier : BELA. \\ PHOTOBELA \\ L'aperçu du dispositif
externe de BELA présenté ci-dessus témoigne notamment de la présence
d'entrées analogiques pour connecter des instruments électriques ou
enregistreurs sonores. On comprend alors que le rôle de BELA au sein
de ce projet est l'utilisation d'un programme similaire à celui
implémenté par les étudiants de l'ENSEIRB en temps réel par un groupe
d'improvisateurs, dont les instruments seraient connectés à cet
outil. Ce dernier dispose déjà de fonctions de traitement du son
codées en C++, mais revêt surtout un intérêt pour les développeurs,
qui peuvent librement modifier et enrichir son programme.

\subsubsection{Un logiciel d'aide à l'improvisation musicale}
En juin 2017, Jérémy Lixandre, membre des six étudiants ayant amorcé
le projet, poursuit seul ces travaux en s'aidant cette fois de BELA,
au cours d'un stage de deux mois au laboratoire du SCRIME de
Bordeaux. Il reprend le concept de \"matrice de corrélation\" mais
doit reprendre l'implémentation du logiciel à zéro, puisqu'il
abandonne le langage Python utilisé lors des premiers travaux pour
passer à la programmation C++ exigée par le code de BELA. \\
De plus, il relègue les travaux de recherche sur la corrélation et ses
formules, des tâches relevant de la physique et des mathématiques, au
second plan pour mener un projet essentiellement logiciel. En
utilisant le code de BELA, il recrée un outil permettant d'analyser la
corrélation de signaux sonores et d'afficher la matrice de corrélation
présentée plus haut. Seulement cette fois, les signaux sonores traités
proviennent de BELA et peuvent donc être produits par des instruments
branchés directement sur le système. Ce nouveau programme permet donc
à des musiciens d'avoir un retour visuel en temps réel de leur
improvisation, et peut même comparer des pistes mono-instrumentales
jouées en temps réel avec des fichiers sonores numériques déjà
existant. 

\subsubsection{Détails sur le programme existant}
Le programme développé par Jérémy s'intitule \"VisualImpro\". Il se
veut \"générique\", a été implémenté de sorte à permettre à des
développeurs de l'améliorer et de le modifier facilement, et à des
utilisateurs renseignés de modifier certains paramètres et
configurations liés aux calculs de la matrice. Son architecture
logicielle constitue le socle de notre travail, c'est elle que nous
allons devoir ré-organiser et enrichir, afin notamment d'ajouter de
nouvelles fonctionnalités au programme. \\
Le code du programme doit notamment contenir trois fichiers .cpp dont
les noms sont Prepoc*.cpp, Coeff*.cpp et Color*.cpp et qui contiennent
respectivement la fonction de \"pre-processing\" ou traitement du
signal en amont, la fonction de calcul du coefficient de corrélation
et la fonction associant un coefficiant à une couleur. Le programme
est construit de sorte à ce que tout fichier respectant ce nommage
puisse être ajouté au programme afin de permettre à un utilisateur de
choisir la fonction de \textit{pre-processing}/calcul de coefficiant
de corrélation/traduction en couleur de son choix.
\begin{itemize}
  \item La fonction \"Preproc\" prend en entrée une matrice de
    vecteurs représentant les signaux d'entrée. Elle retourne une
    nouvelle matrice de vecteurs, qui pourra présenter des
    échantillons de plus petite taille que la matrice d'entrée par
    exemple, dans un souci d'optimisation des performances du
    programme.
  \item La fonction \"Coeff\" prend en entrée deux vecteurs
    (appartenant à la matrice de sortie précédemment abordée) et
    retourne une valeur comprise entre 0 et 1 et correspondant à la
    corrélation établie entre les deux signaux traités. On peut
    imaginer une infinité potentielle de calculs pour donner lieu à
    une corrélation dans le cadre de ce programme ; il s'agit de
    calculs relativement arbitraires qui seront décrits plus en détail
    dans la suite de ce mémoire.
  \item La fonction \"Color\" prend le coefficient précédemment obtenu
    pour entrée et retourne un triplet RGB ; il s'agit d'un objet C++
    défini par l'une des classes du programme.
\end{itemize}
Ces trois fichiers sont répertoriés dans un dossier \"process\". Un
fichier de configuration permet d'écrire quel fichier choisir pour
chacune des trois fonctions. \\

Le seul fichier imposé par l'IDE de Bela se nomme \"render.cpp\". Il
contient lui-même trois fonctions :
\begin{itemize}
\item La fonction setup() initialise et prépare les ressources de
  traitement du son.
\item La fonction render() s'appelle de manière régulière et répétée
  tout au long du processus audio. Elle a pour arguments des buffers
  contenant les échantillons à traiter.
  \item La fonction cleanup() est appelée à la fin du processus pour
    libérer les ressources allouées et mettre fin à certaines tâches.
\end{itemize}

Une structure \"AuxiliaryTask\" est mise à la disposition du
programmeur et est destinée à répertorier des fonctions trop coûteuses
en temps à l'exécution pour le code de render.cpp. \\

Le fichier main.cpp, en plus de lancer le programme, fait office de
fichier de configuration. L'utilisateur peut y entrer les noms des
fonctions de \textit{pre-processing}/calcul de coefficient/calcul de
triplet RGB de son choix. D'autres configurations purement relatives
au traitement de l'audio peuvent être modifiées dans le fichier
render.cpp.

\subsubsection{Motivation, intérêt et avenir du programme VisualImpro}


\end{document}
