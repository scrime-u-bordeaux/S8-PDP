\section{Refactoring et révision de l'interface graphique}

\subsection{Révisions générales sur le code}
  - commentaires d'en-tête à chaque début de fichier source (.cpp) pour
  expliquer l'intérêt de chacun des fichiers.
  - Refactoring intégral du main.cpp : transformation d'un gros bloc d'une seule
  fonction main en découpage fonctionnel. Le main appelle la fonction launch,
  qui elle-même appelle d'autres fonctions et ainsi de suite si besoin.
  - Refactoring du render.cpp : Découpage fonctionnel de la fonction setup
  en plusieurs fonctions regroupées par utilisation. Le refactoring de la
  fonction render n'était pas possible de la même manière car la fonction render
  exécute une boucle de traitement audio, en créant des taches auxiliaires
  exécutées par d'autres threads pour garantir un traitement rapide (qui doit
  être au minimum plus rapide que la vitesse de lecture des pistes), et le
  découpage fonctionnel de cette fonction n'était pas possible sans créer de
  conflits sur les variables enregistrant le nombre de tours de boucle ou les
  variables contenant les signaux audio par exemple.
  - Ajout de commentaires en masse : Le projet m'ayant paru incompréhensible au
  premier abord, il fallait commenter d'avantage l'utilisation des fonctions
  principales permettant de comprendre les enjeux du programme simpelement en
  lisant le programme (et non pas en essayant de comprendre en allant chercher
  sur internet). Les fichiers principaux, main.cpp et render.cpp, ont donc
  beaucoup été commentés.
  - Séparation des fichiers de traitement (dossier process/) par fonctions :
  les dossiers sont Coeff, Color, Preproc, Mix, contenant chacun les fichiers
  correspondant au type de leur fonction de traitement.
  - Création des fichiers de tests et séparation par type de test comme pour
  le dossier process, avec des dossiers TestCoeff, TestColor, TestPreproc et
  TestMix, contenant chacun les tests des fonctions correspondant à leur type de
  traitement. Jérémy n'avait fait aucun test alors on lui a testé ses fonctions
  à cet enculé.
  - Séparation des dossiers généraux : ___PAS ENCORE ETABLI A 100%___ mais genre
  dans l'idée il faudra découper les fichiers comme le prof l'avait dit genre
  avec un dossier src contenant les fichiers source, un dossiers include avec
  les header (.hpp), un dossier process qu'on a déjà, un dossier test qu'on a
  déjà aussi, mais à voir comment les organiser et il faudra adapter les
  makefiles pour ça aussi.
  - Arrangement des include par type d'include et par ordre alphabétique, et
  remise en forme du code pour garder un style cohérent selon les différents
  fichiers, au niveau de la forme et l'alignement des commentaires,
  les indentations, les crochets au même endroit, et renommage des variables
  pour une meilleure compréhension.

\subsection{Les changements apportés à l'interface graphique}
