\newpage
\section{Conclusion}
\subsection{Éléments manquants du projet}
\paragraph{}
Parmi les demandes de nos clients, il nous était demandé de fournir une nouvelle fonction de calcul de corrélation, basée sur l'auto-corrélation temporelle, qui pourrait retourner non plus un coefficient de corrélation, mais pour chaque paire d'entrées, un décalage auquel la paire est la plus corrélée. Seulement cette fonction de calcul a suscité chez nous quelques interrogations :
\begin{itemize}
    \item Comment représenter visuellement un décalage temporel?
    \item Est-ce que la complexité d'une telle fonction ne créerait pas trop de latences?
    \item Si l'échelle temporelle devait être plus grande, faudrait-il utiliser une file contenant les tampons sur lesquels il faut lancer le calcul?
\end{itemize}
\paragraph{}
De ce fait, nous avons revu la priorité de cette demande à la baisse, et finalement, nous n'avons pas eu le temps de mener cet objectif à bien.
\paragraph{}
De plus, l'outil VisualImpro avait pour objectif de pouvoir ajouter des effets aux entrées. Dès le départ, la priorité de ce besoin était basse, nous avions prévu de ne pas s'attarder sur cet aspect là.
\paragraph{}
Nous avions l'intention de tester Bela avec des musiciens (en "live"), mais nos tests de l'outil se sont limités à l'utilisation de fichiers \verb!.wav! triviaux, afin de prédire les retours et tester la fiabilité de notre programme.
\subsection{Évolution de l'outil}
\paragraph{}
Dans une future itération de ce projet, on pourrait imaginer un changement des paramètres en pleine exécution au lieu de seulement les fixer avant. De plus, tous les éléments cités précédemment pourront être implémentés.
\paragraph{}
Une des raisons pour lesquelles nous avons opté pour un affichage graphique avec QT, c'est qu'à partir de ce code, il serait facile d'imaginer un portage de l'affichage sur mobile, dans un soucis de confort pour les musiciens.
