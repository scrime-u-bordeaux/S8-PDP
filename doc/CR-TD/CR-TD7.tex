\documentclass[a4paper]{article}

%% Language and font encodings
\usepackage[french]{babel}
\usepackage[utf8x]{inputenc}
\usepackage[T1]{fontenc}

%% Sets page size and margins
\usepackage[a4paper,top=3cm,bottom=2cm,left=3cm,right=3cm,marginparwidth=1.75cm]{geometry}

%% Useful packages
\usepackage{amsmath}
\usepackage{graphicx}
\usepackage[colorinlistoftodos]{todonotes}
\usepackage[colorlinks=true, allcolors=blue]{hyperref}

\title{Projet de Programmation : compte-rendu TD7}
\author{Alexandre Casanova--Franger\and Gauthier Lamarque\and Lucas Vivas\and Paul Simorre}
\begin{document}
\maketitle

\section{Conseils pour la rédaction du mémoire}

\begin{itemize}

	\item Commencer le mémoire par un résumé tel que décrit dans les slides du cours
	\item Supprimer les passages sur l'historique ou le déroulement du projet
	\item Citer le nom complet de Jérémy Lixandre s'il est nécessaire de le citer
	\item Ajouter la liste des besoins avec leurs priorités dans l'introduction
	\item Axer le mémoire sur les besoins
	\item Envoyer une version corrigée du mémoire en début de semaine prochaine
	\item Supprimer le diagramme de Gantt
	\item Evoquer les besoins qui n'ont pas été réalisés mais auxquels nous avons pensés
	      et qui sont potentiellement réalisables par un futur groupe de projet (menu de
	      configuration sur page web, affichage sur smartphone etc.)

\end{itemize}

\section{Conseils pour le rendu}

\begin{itemize}

	\item Ordonner le dépôt systématiquement, la structure du dépôt soit refléter une
	      hiérarchie logique
	\item Assurer la logique des noms de fichiers, de l'architecture
	\item Se demander comment le programme va évoluer à l'avenir pour répondre à des
	      problèmes comme doit-on modifier et déplacer la traduction en triplet RGB de la
	      couche métier à la couche présentation ?
	\item Attention aux conventions lors du refactoring : les \verb!.hpp! doivent être placés après les autres \verb!include! en début de fichier
	\item Les commentaires doivent être écrits en une seule langue
	\item Mieux vaut marquer les noms de tous les membres de l'équipe de projet dans les codes
	\item Eviter les enter-score dans les noms de variables/données membres
	\item Envisager de refactorer le main, trop chargé
	\item Multiplier les fonctions de corrélation pour les tests

\end{itemize}




\section{Conseil pour la soutenance}

\begin{itemize}

	\item Envisager de parler des besoins avant de parler de l'existant, contrairement à ce qui a été fait dans le mémoire, afin de situer immédiatement l'intérêt du projet
\end{itemize}
\end{document}
