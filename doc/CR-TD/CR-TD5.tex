\documentclass[a4paper]{article}

%% Language and font encodings
\usepackage[french]{babel}
\usepackage[utf8x]{inputenc}
\usepackage[T1]{fontenc}

%% Sets page size and margins
\usepackage[a4paper,top=3cm,bottom=2cm,left=3cm,right=3cm,marginparwidth=1.75cm]{geometry}

%% Useful packages
\usepackage{amsmath}
\usepackage{graphicx}
\usepackage[colorinlistoftodos]{todonotes}
\usepackage[colorlinks=true, allcolors=blue]{hyperref}

\title{Projet de Programmation : compte-rendu TD6}
\author{Alexandre Casanova--Franger\and Gauthier Lamarque\and Lucas Vivas\and Paul Simorre}
\begin{document}
\maketitle

\section{Généralités abordées sur l'avenir du projet}

\begin{itemize}

\item Achever l'introduction mémoire : les informations réunies dans les compte-rendus de TD et nos connaissances sur le projet nous fournissent le matériel nécessaire.
\item Reporter les schémas (et leurs versions améliorées) dans le mémoire, cela facilitera grandement la compréhension du sujet et du projet.
\item En ce qui concerne la fonction de corrélation qu'il nous reste à implémenter, il faudra être précis pour "démystifier" le côté philosophique de la corrélation, et ne se concentrer que sur le traitement concret des entrées.

\item Être précis dans le rapport : Bien montrer où l'on agit par rapport au cycle de la fonction principale de Bela.

\item Bien indiquer où le refactoring a été fait avec la comparaison avant/après, de manière à pouvoir analyser les changements de manière claire et ainsi pouvoir remarquer les améliorations apportées au programme. Cela impliquerait notamment de différencier le refactoring à l'intérieur du Bela, du refactoring visuel en sortie (couche présentation/couche métier). On prendra donc soin de garder le code de notre prédécesseur pour montrer la différence avant/après de l'état du programme.

\item Montrer et prouver la généricité de la sortie.

\item Il faudra indiquer que nous avons tenté de répondre aux besoins des clients uniquement dans nos travaux. Comme le sujet peut amener à un grand nombre de nouveaux objectifs à atteindre, il faut bien cerner ce qui nous a été demandé et nous focaliser là-dessus. De même pour la soutenance, il faut que nous nous défendions par rapport au cahier des besoins si l'on nous demande pourquoi nous n'avons pas implémenté une autre fonctionnalité quelconque à la place de nos travaux.

\item Il faudrait des tests concrets pour appuyer nos propos, dont on connaît le résultat à l'avance. Cela permettrait une meilleure compréhension au jury. Pour cela, on effectuera nos tests avec des données triviales en entrée (de type sinusoïde) dont on connaît les résultats post-traitement.

\item Pour la création de test, il faudra que l'on se pose les bonnes questions. Qu'est-ce que l'on cherche à tester ? Quelles données allons-nous utiliser ? Faire des scénarios de test, de l'injection d'une entrée sonore à la sortie visuelle et audio.

\item Pour l'intérêt du projet, il faudrait essayer d'évaluer en nombre d'heures le reste des tâches à accomplir pour pouvoir ainsi définir un planning pour effectuer ces travaux.
\end{itemize}

\end{document}