\documentclass[a4paper]{article}
\usepackage[english]{babel}
\usepackage[utf8x]{inputenc}
\usepackage[T1]{fontenc}
\usepackage[a4paper,top=3cm,bottom=2cm,left=3cm,right=3cm,marginparwidth=1.75cm]{geometry}
\usepackage{amsmath}
\usepackage{graphicx}
\usepackage[colorinlistoftodos]{todonotes}
\usepackage[colorlinks=true, allcolors=blue]{hyperref}

\title{Projet de Programmation : Compte-rendu de TD n°4}
\author{Lucas Vivas, Paul Simorre, Gautier Lamarque et Alexandre Casanova--Franger}

\begin{document}
\maketitle

\paragraph{}
Nous aurions dû mieux nous renseigner sur l'existant et observer le code à l'avance
car il possède une mauvaise architecture. Nous devrons donc nécessairement passer
par une phase de refactoring plus ou moins importante et nous pouvons ajouter cela
au cahier des besoins. De plus nous devons maintenant expliquer le code fourni
pour le fonctionnement de Bela, en précisant le main, le render, et tout autre
gros fichier important et aussi la façon dont nous allons les modifier modulairement.
Préciser l'importance des tâches auxiliaires (propres à Bela) dans notre
projet : il faudrait en faire un slide pour le powerpoint de l'audit. Passer rapidement
sur l'utilité des variables globales.

\paragraph{}
Pour notre 1ère release, il est nécessaire de penser à l'interface graphique, au
refactoring, à l'explication du code, au squelette de nos fonctions et fichiers,
et aux premiers tests. Il ne faudra pas forcément arriver à un résultat très
sophistiqué mais il faudra être très précis dans nos explications, notamment expliquer
comment notre code entre en jeu et la manière dont Bela l'utilisera. Il faudrait
que nos ajouts aient peu de dépendance avec le reste du programme pour qu'ils puissent
être exploités par nos successeurs sans trop d'obligations techniques, notamment
par rapport à l'ajout d'effets. Nous commenterons au maximum notre code et mettrons
en place une architecture compréhensible et réutilisable pour que nos potentiels successeurs
aient le moins de mal à reprendre notre code et se l'approprier.

\paragraph{}
Pour notre savoir personnel et avec possibilité de l'utiliser dans notre projet,
il faudrait que l'on se renseigne sur les lambdas-expressions en C++ (11). Se
renseigner également sur les "foncteurs" (surcharge des opérateurs \verb!()!,
qui semblent avoir un lien avec les lambda-expressions. Enfin, pour la suite de
notre projet, on pourra privilégier une programmation modulaire plutôt qu'objet,
avec composition pour le réglage des volumes, ce qui nous permettrait par la suite
avec l'ajout des effets, d'avoir le choix d'attacher certains modules au son selon
le choix de l'utilisateur.

\end{document}
