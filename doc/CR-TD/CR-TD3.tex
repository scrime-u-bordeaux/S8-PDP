\documentclass[a4paper]{article}
\usepackage[english]{babel}
\usepackage[utf8x]{inputenc}
\usepackage[T1]{fontenc}
\usepackage[a4paper,top=3cm,bottom=2cm,left=3cm,right=3cm,marginparwidth=1.75cm]{geometry}
\usepackage{amsmath}
\usepackage{graphicx}
\usepackage[colorinlistoftodos]{todonotes}
\usepackage[colorlinks=true, allcolors=blue]{hyperref}

\title{Projet de Programmation : Compte-rendu de TD n°3}
\author{Lucas Vivas, Paul Simorre, Gautier Lamarque et Alexandre Casanova--Franger}

\begin{document}
\maketitle

\section*{Remarques sur le cahier des besoins}
\paragraph{}
Comme nous sommes la troisième itération du projet, la partie existante est très
importante. Donc il faut préciser celle-ci et l'étayer.
Ensuite, il faut bien distinguer ce qui doit figurer dans l'introduction et
dans la description de l'existant.\\
Dans l'introduction, nous devons retrouver tout ce qui a attrait au projet dans
son ensemble (nous avons parlé de la "boucle principale").\\
Dans la description de l'existant par contre, on doit retrouver tous les
éléments liés au programme.\\
Afin que tout cela soit bien clair, il serait bien de l'illustrer à l'aide de
schémas simples.
Tout ce qui concerne l'affichage de la matrice (c'est-à-dire la création du
serveur web avec NodeJS et le code HTML pour l'affichage) a été éludé, donc il
faut absolument en parler dans la description de l'existant. Son apparition sur
le diagramme de déploiement a destabilisé les lecteurs.\\
Nous devons aussi rajouter dans les références le lieu de stage de notre
prédécesseur.

\section*{Pistes pour la 1e release}
\paragraph{}
Les tests seront effectués à partir de fichiers numériques. \\
La rédaction du mémoire débute en même temps que le développement.\\
Nous avons réfléchi à la pertinence de l'affichage sur navigateur, et il nous
semble plus cohérent de créer une interface graphique dans le même langage
que le programme (C++).
Enfin, nous allons réfléchir à un refactoring du code existant afin d'élargir
la généricité.
\end{document}
