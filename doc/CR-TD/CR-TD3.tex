\documentclass[a4paper]{article}
\usepackage[english]{babel}
\usepackage[utf8x]{inputenc}
\usepackage[T1]{fontenc}
\usepackage[a4paper,top=3cm,bottom=2cm,left=3cm,right=3cm,marginparwidth=1.75cm]{geometry}
\usepackage{amsmath}
\usepackage{graphicx}
\usepackage[colorinlistoftodos]{todonotes}
\usepackage[colorlinks=true, allcolors=blue]{hyperref}

\title{Projet de Programmation : Compte-rendu de TD n°3}
\author{Lucas Vivas, Paul Simorre, Gautier Lamarque et Alexandre Casanova--Franger}

\begin{document}
\maketitle

\section*{Remarques sur le cahier des besoins}
\paragraph{}
Comme nous sommes à la troisième itération du projet, la partie existante est très
importante. Donc il faut préciser celle-ci et l'étayer. Les remarques suivantes
pourront être appliquées pour enrichir le cahier des besoins en vue de l'audit,
mais aussi et surtout pour la rédaction du mémoire, qui doit être commencée dès à
présent et devra se poursuivre tout au long du déroulement du projet. \\
Il est important de parler des travaux qui ont déjà été menés par d'autres sur ce
projet commun : par qui, dans quel cadre, quelle équipe a fait quoi quel
lieux de stage etc. Il est important de détailler dans l'introduction la façon
dont nous allons intervenir pour améliorer l'existant. \\
Il faut bien distinguer ce qui doit figurer dans l'introduction et
dans la description de l'existant.\\
Dans l'introduction, nous devons retrouver tout ce qui a trait au projet dans
son ensemble (nous avons parlé de la "boucle principale" : ce qui part des
musiciens et revient aux musiciens). Il est essentiel de décrire la notion de 
corrélation est d'expliquer en quoi elle est centrale au projet. Il faut veiller à
bien lier les notions d'improvisation et de corrélation en justifiant sa généricité. \\
Dans la description de l'existant par contre, on doit retrouver tous les
éléments liés au programme. Il faut clarifier certains points, comme le rôle du web
dans l'implémentation de notre prédécesseur. \\
Afin que tout cela soit bien clair, il serait bien de l'illustrer à l'aide de
schémas simples. Plutôt que les schémas techniques vus en cours tels que l'UML,
on pourra utiliser un schéma simplifié du dispositif général encore dans le but
d'être le plus clair possible. \\
Les schémas devront également illustrer de façon claire la façon dont nous allons nous-
mêmes participer au projet ; la construction de tels schémas aura pour avantage de faire
surgir des besoins que nous aurions pu oubier. L'utilisation doit être clarifiée, il
faut faire des choix : confier le programme à un opérateur qui étudie les retours
visuels ou montrer ces retours en temps réel aux musiciens par exemple. \\
Les besoins évoqués dans le cahier doivent être mieux décrits dans le mémoire,
notamment vis-à-vis de la façon dont nous allons les remplir. L'extension de la 
généricité du code doit être évoquée dans les besoins non-fonctionnels. Des détails
techniques doivent être expliqués en priorité, et avant tout appréhendés par l'équipe
elle-même, par exemple au sujet du stockage éventuel/enregistrement des entrées sonores.


\section*{Pistes pour la 1e release}
\paragraph{}
Tout ce qui concerne l'affichage de la matrice (c'est-à-dire la création du
serveur web avec NodeJS et le code HTML pour l'affichage) a été éludé, donc il
faut absolument en parler dans la description de l'existant. Son apparition sur
le diagramme de déploiement a déstabilisé les lecteurs. Le détail du rôle et du
fonctionnement de cette matrice doit apparaître dans nos travaux écrits. \\
L'affichage de cette matrice, quant à lui, doit être repensé car le choix de 
NodeJS + HTML n'a pas été justifié par notre prédécesseur et semble peu rationnel.
Nous pensons nous tourner vers une interface graphique implémentée par le langage
C++, notre principal outil de travail pour ce projet ; il s'agirait d'un travail de
refactoring qui pourrait demander un certain temps, tout comme l'extension de la
généricité du code permettant les fonctions de corrélation plus complexes suggérées par
le client qui a été évoqué. Ces nouveaux travaux d'approche objet nous contraindront à
revoir à la baisse le diagramme de Gantt. \\
Les tests seront effectués à partir de fichiers numériques. Nous utiliserons des
fonctions triviales pour les réaliser. \\
Enfin, nous allons réfléchir à un refactoring du code existant afin d'élargir
la généricité.
\end{document}
