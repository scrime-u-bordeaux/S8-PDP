\documentclass{article}

\usepackage[utf8]{inputenc} % accents
\usepackage[T1]{fontenc}      % caractères français
\usepackage{geometry}         % marges
\usepackage[francais]{babel}  % langue
\usepackage{graphicx}         % images
\usepackage{verbatim}

\title{Document d'analyse des besoins}
\author{Alexandre Casanova-Franger\\
        \and
        Gauthier Lamarque\\
        \and
        Paul Simorre\\
        \and
        Lucas Vivas\\}

\begin{document}
  \maketitle
  \section{Introduction}
    \paragraph{}
    Ce projet consiste à réaliser un outil informatique d'analyse musicale dans
    l'optique de faciliter l'improvisation. À terme, cet outil aura pour
    objectif de permettre à des musiciens de pouvoir se calibrer sur les membres
    qui jouent le mieux ensemble. De plus, cet outil aura une portée
    scientifique, dans le sens où il sera possible d'étudier l'adaptation des
    musiciens grâce au retours fournis par l'outil. \\
    Cet outil est basé sur une plateforme externe appelée Bela, qui est un
    système embarqué de traitement audio en temps réel.
  \section{Description et analyse de l'existant}
    \paragraph{}
    À ce jour, l'outil est capable de fournir une représentation graphique de la
    corrélation des entrées audio. Ces entrées sont représentées par des
    tableaux, et ensuite il existe une fonction prenant ces tableaux en entrée
    et qui retourne un nombre flottant compris entre 0 et 1, qui représente la
    corrélation entre les deux entrées. Pour n entrées, la représentation sera
    donc une matrice carrée de taille n.\\
    Dans un souci d'évolution, il est possible d'ajouter de nouvelles fonctions
    de calcul de corrélation. Il suffit de respecter une signature de méthode,
    et de placer les fichiers sources dans le dossier prévu à cet effet.\\
  \section{Description des besoins}
    \subsection{Besoins fonctionnels}
    \paragraph{}
    \begin{itemize}
      \item L'utilisateur pourra avoir un retour sonore qui dépendra de la
      matrice de corrélation et d'une configuration,
      \item L'utilisateur pourra avoir une représentation graphique des niveaux
      sonores sous la forme d'une matrice, de la même manière que la matrice de
      corrélation.
      \item L'utilisateur aura à sa disposition une liste de configurations
      pré-établies, dont les suivantes :
      \begin{itemize}
        \item Les entrées les plus corrélées sont plus fortes,
        \item Les entrées les moins corrélées sont plus fortes.
      \end{itemize}
      \item L'utilisateur pourra ajouter une nouvelle configuration, suivant
      une signature donnée.
      \item L'utilisateur pourra changer la fenêtre temporelle de calcul de
      corrélation.
    \end{itemize}
    \subsection{Besoins non-fonctionnels}
    \begin{itemize}
      \item Les signatures des méthodes calculant les niveaux sonores des
      entrées devront être génériques de façon à permettre aux utilisateurs d'en
      ajouter de nouvelles.
    \end{itemize}
    \subsection{Scénarios, prototypes, diagrammes, etc.}
  \section{Diagramme de Gantt}
  \section{Bibliographie}
\end{document}
