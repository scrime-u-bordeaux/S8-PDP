\documentclass{article}

\usepackage[utf8]{inputenc} % accents
\usepackage[T1]{fontenc}      % caractères français
\usepackage{geometry}         % marges
\usepackage[francais]{babel}  % langue
\usepackage{graphicx}         % images
\usepackage{verbatim}
\usepackage{url}
\bibliographystyle{alpha}

\title{Document d'analyse des besoins}
\author{Alexandre Casanova-Franger\\
        \and
        Gauthier Lamarque\\
        \and
        Paul Simorre\\
        \and
        Lucas Vivas\\}

\begin{document}
\maketitle
\section{Introduction}
\paragraph{}
Ce projet consiste à améliorer un outil informatique d'analyse sonore dans
l'optique d'approfondir notre compréhension de l'improvisation musicale. À
terme, cet outil aura pour objectif de permettre à des musiciens d'avoir un
aperçu discret (donc quantifié) de leur improvisation. De plus, cet outil aura
une portée scientifique, dans le sens où il sera possible d'étudier l'adaptation
des musiciens grâce aux retours fournis par l'outil. \\
Cet outil est basé sur une plateforme externe appelée Bela \cite{BELA},
qui est un système embarqué de traitement audio en temps réel.
\paragraph{}
À l'origine, six étudiants de l'ENSEIRB-MATMECA ont réalisé un projet similaire
en début d'année 2017, mais ils n'ont pas utilisé le système embarqué Bela. Ils
ont construit un programme en Python, et ont géré les entrées/sorties audio
grâce à des librairies externes. Dans les mois suivants, un de ces six
étudiants, Jérémy Lixandre, a réalisé un stage de deux mois au SCRIME (Studio
de Création et de Recherche en Informatique et Musiques Expérimentales) et a
recommencé le même projet, mais cette fois en se basant sur Bela et son langage
de programmation, le C++. Notre projet est donc la poursuite du travail réalisé
par Jérémy Lixandre durant son stage.
\section{Description et analyse de l'existant}
\paragraph{}
À ce jour, l'outil est capable de fournir une représentation graphique de la
corrélation des entrées audio. La corrélation est une notion centrale
concernant cet outil. La corrélation est en fait une relation statistique
entre deux ou plusieurs variables, et dans notre cas, il s'agit de la
relation entre deux échantillons audio.
\paragraph{}
Les entrées audio sont gérées par le système embarqué Bela, qui peut
recevoir 8 entrées analogiques (par l'intermédiaire de micros), 2 entrées
audio (par l'intermédiaire de prises Jack) et enfin la possibilité d'inclure
des fichiers audio (de type wav).\\
La mécanique principale du programme consiste à récupérer des échantillons
provenant des entrées audio sur un temps donné ($\Delta t$), de les stocker dans
un tableau de vecteurs où chaque vecteur (de flottants) représente le signal d'une des entrées.
Ensuite, le programme appliquera sur ces signaux un pré-traitement qui permettra
de modifier les vecteurs afin de pouvoir baser le calcul de corrélation sur différentes
caractéristiques de l'onde. Par exemple, le programme est capable de calculer
l'enveloppe d’énergie d'une onde afin que le calcul de corrélation se fasse en fonction
de cette enveloppe qui permettrait de comparer par exemple les timbres des instruments.
Ensuite il applique une fonction de calcul de corrélation sur chaque paire d'échantillons.
Cette fonction peut changer et permettra de comparer les vecteurs de différentes manière.
\\
Le calcul de la corrélation prend en entrée deux vecteurs de
nombres flottants et renvoie un nombre flottant compris entre 0 et 1.
Dans l'état actuel de l'outil, le calcul de la corrélation correspond au
produit scalaire des deux vecteurs de flottants passés en entrée. Cependant,
grâce au fichier de configuration de l'outil, il est possible de modifier la
méthode de calcul de cette corrélation.\\
Face à la complexité que représente la création d'un tel outil, notre
prédécesseur a fait le choix d'implémenter une version simplifiée du calcul
de corrélation.
\paragraph{}
Enfin, une fois que les coefficients de corrélation sont calculés, ceux-ci sont
affichés suivant une matrice de corrélation, où $x_{i,j}$ correspond au
coefficient de corrélation entre l'entrée i et l'entrée j.\\
Sachant cela, la matrice de corrélation obtenue est donc une matrice
symétrique (où $x_{i,j} = x_{j,i}$) et la diagonale de celle-ci est égale à
1 ($x_{i,i} = 1$).
\paragraph{}
Un autre élément central de cet outil est le fichier de configuration
contenant les paramètres nécéssaires au bon fonctionnement du programme. Les
paramètres à renseigner ou à modifier sont les suivants:
\begin{itemize}
 \item L'activation des effets et la taille des buffers correspondants
       (toutefois, l'ajout d'effets semble diminuer les performances du
       programme),
 \item Le nombre d'entrées analogiques que l'on souhaite activer (de 0 à 8),
 \item Le nombre d'entrées audio que l'on souhaite activer (de 0 à 2),
 \item Les chemins relatifs vers des fichiers .wav que l'on souhaiterait
       ajouter en entrée,
 \item Le nom du fichier (sans l'extension \verb!.cpp!) contenant la
       fonction de conversion du coefficient de corrélation en triplet RGB,
 \item Le nom du fichier (sans l'extension \verb!.cpp!) contenant la
       fonction de calcul du coefficient de corrélation,
 \item Le nom du fichier (sans l'extension \verb!.cpp!) contenant la
       fonction de pré-traitement des entrées.
\end{itemize}
\begin{figure}[h]
 \caption{\label{bela_process}Déroulement du programme existant}
 \centering
 \includegraphics[scale=0.40]{bela_process.jpg}
\end{figure}
\newpage
\section{Description des besoins}
\subsection{Besoins fonctionnels}
\paragraph{}
\begin{itemize}
 \item L'utilisateur pourra avoir un retour sonore, où les volumes des
       entrées seront modifiés selon un certain mixage (qui dépendra de la
       corrélation des entrées),
       \paragraph{}
       Il s'agit là de l'objectif principal de ce projet. L'utilisateur pourra
       obtenir un retour audio sur la sortie correspondante du système embarqué.
       Cependant, les volumes des différentes entrées seront modifiés, il y
       aura une étape de mixage pour modifier ces volumes, elle sera explicitée
       lors des prochains points.
       \\
       Afin de vérifier le bon fonctionnement ce retour audio, nous allons tester
       différents cas :
       \begin{itemize}
        \item Le cas limite où nous prenons un signal aléatoire (de moyenne égale à 0 mais de variance aléatoire)
              que nous dédoublons, avec
              cela le coefficient de corrélation, peu importe les prétraitements ou la fonction de
              corrélation utilisée, sera égal à 1.
        \item Le cas pour tester la cohérence serait de prendre une entrée aléatoire
              qu'on dédouble puis à prendre le double et de le décaler au fur et à mesure dans le temps
              afin de voir si la corrélation diminue ou non.
        \item Un autre test positif serait, comme pour les tests précédents, de prendre deux fois la même
              entrée aléatoire puis de rajouter un bruit sur le doublon puis d'augmenter
              ce bruit afin d'observer la diminution de la corrélation.
        \item Enfin un dernier test de limite serait de prendre un signal aléatoire
              et un autre signal muet (de variance et de moyenne égale à 0) et le résultat
              devrait être un coefficient de corrélation égale à 0.
       \end{itemize}
       \\
       
 \item L'utilisateur pourra choisir une configuration selon laquelle un
       vecteur de mixage sera créé,
       \paragraph{}
       Le mixage sera implémenté comme une fonction qui prendra en entrée la
       matrice contenant les coefficients de corrélation, et renverra en sortie
       un vecteur contenant les volumes attribués à chaque entrée.\\
       
 \item L'utilisateur pourra ajouter une fonction de mixage, qu'il précisera
       dans le fichier de configuration,
       \paragraph{}
       Comme pour les étapes de pré-traitement, de calcul de corrélation, et de
       conversion vers un triplet RGB, il sera possible pour l'utilisateur de
       préciser dans le fichier de configuration quel fichier contient la
       fonction permettant d'obtenir le mixage souhaité. Nous avons évoqué avec
       les clients différents exemples de mixage, dont les suivants :
       \begin{itemize}
        \item Augmenter le volume des paires d'entrées les plus corrélées,
        \item Augmenter le volume des paires d'entrées les moins corrélées,
        \item Augmenter le volume des entrées dont la somme des coefficients de
              corrélation avec toutes les autres entrées est la plus élevée.
       \end{itemize}
       \paragraph{}
       
 \item L'utilisateur pourra observer ce vecteur de mixage sur la même page
       web où figure la matrice contenant les coefficients de corrélation,
       \paragraph{}
       Sachant que les utilisateurs de l'outil ne sont pas des développeurs de
       métier, un affichage graphique reste le meilleur moyen de représenter les
       informations importantes. La matrice contenant les coefficients de
       corrélation est représentée par une matrice de carrés colorés, et le
       code couleur est fourni par une fonction située dans un fichier dont le
       nom est précisé dans le fichier de configuration. De la même façon, le
       vecteur de mixage sera représenté par des carrés colorés, et le
       pourcentage du volume sera indiqué sous ces carrés.\\
       
 \item L'utilisateur pourra changer les paramètres d'utilisation directement sur
       l'interface web et non en passant par le fichier de configuration.
       \paragraph{}
       Soit tous les paramètres du fichier de configuration qui sont la taille des buffers
       des échantillons, mettre des effets, et régler le nombre d'entrées analogiques.
       
 \item l'utilisateur pourra sélectionner la fonction de corrélation temporelle.
       \paragraph{}
       Actuellement, l'unique fonction de calcul de corrélation existante est
       la fonction de corrélation statique qui va calculer un coefficient de corrélation
       en fonction de tout les points de l’échantillon. Cette nouvelle fonction
       calculera, à l'aide de transformée de Fourier, le temps ($\tau$) de latence
       entre deux signaux.
       
       
\end{itemize}
\subsection{Besoins non-fonctionnels}
\begin{itemize}
 \item La fonction réalisant le mixage devra être générique.
       \paragraph{}
       Comme dit plus haut, il existe déjà trois étapes du programme qui
       nécessitent des fonctions génériques, et ces fonctions (et leur fichier
       correspondant) sont précisées dans le fichier de configuration (à
       modifier avant l'éxécution du programme). De la même manière, la
       fonction réalisant le mixage des entrées devra respecter les mêmes
       conditions.
 \item L'ajout d'une fonction de mixage ne devra pas créer de latences.
       \paragraph{}
       Dans le rapport émis par le précédent développeur, il est indiqué qu'à
       partir de 15 entrées, ou avec l'ajout d'effets, des latences sont
       visibles. Actuellement, le rafraîchissement de la matrice est de l'ordre
       de la demi-seconde dans son fonctionnement normal. De ce fait, l'étape
       de calcul du mixage en sortie ne devra pas être trop lourde en terme de
       temps de calcul, sous peine d'entraîner de fortes latences à l'affichage.
\end{itemize}
\begin{figure}[h]
 \caption{\label{diag_deploiement}Diagramme de déploiement}
 \centering
 \includegraphics[scale=0.30]{diag_deploiement.jpg}
\end{figure}
\section{Scénario d'utilisation}
\paragraph{}
Nous allons décrire ci-après un scénario d'utilisation. Nous imaginons une
configuration simple pour un scénario réalisable avec une version du projet où
les besoins les plus essentiels aux yeux des clients ont été remplis.
\begin{itemize}
 \item Trois musiciens jouent et s'enregistrent ensemble
 \item Ils veulent obtenir un retour sonore particulier : une version de leur
       morceau où la paire de musiciens la plus corrélée s'entend plus fort que le
       troisième musicien
 \item Ils utilisent une version basique du logiciel et l'on suppose donc que
       les paramètres tels que la fonction de pré-traitement des entrées, la fonction
       de calcul du coefficient de corrélation et celle calculant sa conversion en
       triplet RGB auront été choisis par les développeurs
\end{itemize}
Les musiciens improvisent ensemble en tentant de s'accorder les uns avec les
autres. Grâce au dispositif BELA, chaque instrument est enregistré sur une
piste mono-instrumentale isolée. Le logiciel existant traite les données de
leur enregistrement pour fournir la matrice de corrélation ; grâce à ce retour
visuel, le groupe sait déjà quelle paire de musiciens est la plus corrélée à
un instant t donné.
\begin{figure}[h]
 \centering
 \includegraphics[scale=0.50]{matrice_correlation.png}
\end{figure}
\paragraph{}
Plus la couleur aux indices de deux musiciens est proche du vert clair,
meilleure est la corrélation de leur jeu au moment où cette matrice s'affiche.
On peut voir par exemple qu'à l'instant \textit{t}, les musiciens M1 et M2
sont plus corrélés que les autres couples de musiciens ; ils forment la paire
de musiciens et de pistes mono-instrumentales la plus corrélée.
\paragraph{}
Pour "construire" le retour sonore, notre programme mixe l'enregistrement
grâce aux données de cette matrice. À chaque instant \textit{t}, une valeur
est attribuée à chaque paire de musiciens à partir de son indice de
corrélation. La valeur la plus élevée, celle attribuée à la paire la plus
corrélée, sert à produire un retour sonore sous la forme d'une copie de
l'enregistrement initial où les deux pistes mono-instrumentales constituant
cette paire sont augmentées en volume sonore. Cette paire est donc susceptible
de varier tout au long du morceau : à chaque instant \textit{t}, la paire de
musiciens augmentés en volume sonore peut changer.
\paragraph{}
Les musiciens peuvent alors étudier le retour sonore produit par notre logiciel
et le comparer avec l'enregistrement non modifié par le logiciel. On peut
alors imaginer diverses utilités à ce retour sonore :
\begin{itemize}
 \item Les musiciens pourront préférer conserver l'enregistrement modifié plutôt
       que l'original
 \item Si deux musiciens sont plus régulièrement augmentés en volume sonore que
       leur partenaire tout au long du morceau modifié, ce troisième musicien pourra
       corriger son jeu en conséquence
\end{itemize}


\centering
\includegraphics[scale=0.30]{proto_2.png}

\section{Diagramme de Gantt}
\includegraphics[scale=0.35]{DiagrammeAnalyseBesoins.jpg}
\bibliography{analyse-besoins}
\end{document}
