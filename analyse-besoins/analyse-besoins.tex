\documentclass{article}

\usepackage[utf8]{inputenc} % accents
\usepackage[T1]{fontenc}      % caractères français
\usepackage{geometry}         % marges
\usepackage[francais]{babel}  % langue
\usepackage{graphicx}         % images
\usepackage{verbatim}
\usepackage{url}
\bibliographystyle{alpha}

\title{Document d'analyse des besoins}
\author{Alexandre Casanova-Franger\\
        \and
        Gauthier Lamarque\\
        \and
        Paul Simorre\\
        \and
        Lucas Vivas\\}

\begin{document}
\maketitle
\section{Introduction}
\paragraph{}
Ce projet consiste à améliorer un outil informatique d'analyse sonore dans
l'optique de faciliter l'improvisation musicale. À terme, cet outil aura pour
objectif de permettre à des musiciens de pouvoir améliorer leur
improvisation. De plus, cet outil aura une portée scientifique, dans le sens
où il sera possible d'étudier l'adaptation des musiciens grâce aux retours
fournis par l'outil. \\
Cet outil est basé sur une plateforme externe appelée Bela \cite{BELA},
qui est un système embarqué de traitement audio en temps réel.
\section{Description et analyse de l'existant}
\paragraph{}
À ce jour, l'outil est capable de fournir une représentation graphique de la
corrélation des entrées audio. La corrélation est une notion centrale
concernant cet outil. La corrélation est en fait une relation statistique
entre deux ou plusieurs variables, et dans notre cas, il s'agit de la
relation entre deux échantillons audio.
\paragraph{}
Les entrées audio sont gérées par le système embarqué Bela, qui peut
recevoir 8 entrées analogiques (par l'intermédiaire de micros), 2 entrées
audio (par l'intermédiaire de prises Jack) et enfin la possibilité d'inclure
des fichiers audio (de type wav).\\
La mécanique principale du programme consiste à récupérer des échantillons
provenant des entrées audio, de les stocker dans des buffers, et d'appliquer
une fonction de calcul de corrélation sur chaque paire d'échantillons
(échantillons correspondant à des entrées distinctes). Ces échantillons
sonores sont représentés sous la forme de vecteurs de nombres flottants.\\
De ce fait, le calcul de la corrélation prend en entrée deux vecteurs de
nombres flottants et renvoie un nombre flottant compris entre 0 et 1.
Dans l'état actuel de l'outil, le calcul de la corrélation correspond au
produit scalaire des deux vecteurs de flottants passés en entrée. Cependant,
grâce au fichier de configuration de l'outil, il est possible de modifier la
méthode de calcul de cette corrélation.
\paragraph{}
Enfin, une fois les coefficients de corrélation sont calculés, ceux-ci sont
affichés suivant une matrice de corrélation, où $x_{i,j}$ correspond au
coefficient de corrélation entre l'entrée i et l'entrée j.\\
Sachant cela, la matrice de corrélation obtenue est donc une matrice
symétrique (où $x_{i,j} = x_{j,i}$) et la diagonale de celle-ci est égale à
1 ($x_{i,i} = 1$).
\paragraph{}
Un autre élément central de cet outil est le fichier de configuration
contenant les paramètres nécéssaires au bon fonctionnement du programme. Les
paramètres à renseigner ou à modifier sont les suivants:
\begin{itemize}
	\item L'activation des effets et la taille des buffers correspondants
	      (toutefois, l'ajout d'effets semble diminuer les performances du
	      programme),
	\item Le nombre d'entrées analogiques que l'on souhaite activer (de 0 à 8),
	\item Le nombre d'entrées audio que l'on souhaite activer (de 0 à 2),
	\item Les chemins relatifs vers des fichiers .wav que l'on souhaiterait
	      ajouter en entrée,
	\item Le nom du fichier (sans l'extension \verb!.cpp!) contenant la
	      fonction de conversion du coefficient de corrélation en triplet RGB,
	\item Le nom du fichier (sans l'extension \verb!.cpp!) contenant la
	      fonction de calcul du coefficient de corrélation,
	\item Le nom du fichier (sans l'extension \verb!.cpp!) contenant la
	      fonction de pré-traitement des entrées.
\end{itemize}
\section{Description des besoins}
\subsection{Besoins fonctionnels}
\paragraph{}
\begin{itemize}
	\item L'utilisateur pourra avoir un retour sonore, où les volumes des
	      entrées seront modifiés selon un certain mixage (qui dépendra de la
	      corrélation des entrées),
	      \paragraph{}
	      Il s'agit là de l'objectif principal de ce projet. L'utilisateur pourra
	      obtenir un retour audio sur la sortie correspondante du système embarqué.
	      Cependant, les volumes des différentes entrées seront modifiés, il y
	      aura une étape de mixage pour modifier ces volumes, elle sera explicitée
	      lors des prochains points.\\

	\item L'utilisateur pourra choisir une configuration selon laquelle un
	      vecteur de mixage sera créé,
	      \paragraph{}
	      Le mixage sera implémenté comme une fonction qui prendra en entrée la
	      matrice contenant les coefficients de corrélation, et renverra en sortie
	      un vecteur contenant les volumes attribués à chaque entrée.\\

	\item L'utilisateur pourra ajouter une fonction de mixage, qu'il précisera
	      dans le fichier de configuration,
	      \paragraph{}
	      Comme pour les étapes de pré-traitement, de calcul de corrélation, et de
	      conversion vers un triplet RGB, il sera possible pour l'utilisateur de
	      préciser dans le fichier de configuration quel fichier contient la
	      fonction permettant d'obtenir le mixage souhaité. Nous avons évoqué avec
	      les clients différents exemples de mixage, dont les suivants :
	      \begin{itemize}
	      	\item Augmenter le volume des paires d'entrées les plus corrélées,
	      	\item Augmenter le volume des paires d'entrées les moins corrélées,
	      	\item Augmenter le volume des entrées dont la somme des coefficients de
	      	      corrélation avec toutes les autres entrées est la plus élevée.
	      \end{itemize}
	      \paragraph{}

	\item L'utilisateur pourra observer ce vecteur de mixage sur la même page
	      web où figure la matrice contenant les coefficients de corrélation,
	      \paragraph{}
	      Sachant que les utilisateurs de l'outil ne sont pas des développeurs de
	      métier, un affichage graphique reste le meilleur moyen de représenter les
	      informations importantes. La matrice contenant les coefficients de
	      corrélation est représentée par une matrice de carrés colorés, et le
	      code couleur est fourni par une fonction située dans un fichier dont le
	      nom est précisé dans le fichier de configuration. De la même façon, le
	      vecteur de mixage sera représenté par des carrés colorés, et le
	      pourcentage du volume sera indiqué sous ces carrés.\\

	\item L'utilisateur pourra changer la fenêtre temporelle de calcul de
	      corrélation.
\end{itemize}
\subsection{Besoins non-fonctionnels}
\begin{itemize}
	\item La fonction réalisant le mixage devra être générique.
	      \paragraph{}
	      Comme dit plus haut, il existe déjà trois étapes du programme qui
	      nécessitent des fonctions génériques, et ces fonctions (et leur fichier
	      correspondant) sont précisées dans le fichier de configuration (à
	      modifier avant l'éxécution du programme). De la même manière, la
	      fonction réalisant le mixage des entrées devra respecter les mêmes
	      conditions.
	\item L'ajout d'un mixage ne devra pas perturber le déroulement du programme.
	      \paragraph{}
	      Dans le rapport émis par le précédent développeur, il est indiqué qu'à
	      partir de 15 entrées, ou avec l'ajout d'effets, des latences sont
	      visibles. Actuellement, le rafraîchissement de la matrice est de l'ordre
	      de la demi-seconde dans son fonctionnement normal.
\end{itemize}
\subsection{Scénarios, prototypes, diagrammes, etc.}
\paragraph{}
Nous allons décrire ci-après un scénario que nos clients ont imaginé :\\
L'utilisateur utilise l'outil sur un groupe de musiciens. Des micros sont
branchés sur les différentes entrées analogiques et chaque musicien dispose
d'un retour son. L'utilisateur configure le fichier de configuration afin
que les paramètres décrits ci-dessus soient correctement pris en compte
par l'outil. Les musiciens se mettent à improviser comme bon leur semble,
et l'on peut observer en temps réel les corrélations entre les différents
musiciens. On pourra aussi observer une matrice dite de "mixage" qui
affiche les différents poids des paires de musiciens en fonction d'une
configuration par défaut. À tout moment, l'utilisateur peut choisir une
configuration de mixage différente pour que le retour sonore change.\\
L'intérêt est d'étudier comment les musiciens s'adaptent en fonction du
retour sonore, donc en fonction de la configuration de mixage choisie.
\section{Diagramme de Gantt}
\includegraphics[scale=0.35]{DiagrammeAnalyseBesoins.jpg}
\bibliography{analyse-besoins}
\end{document}
